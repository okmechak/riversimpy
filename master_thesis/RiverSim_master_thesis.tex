\documentclass[]{pracamgr}

\begin{document}

\maketitle

%tu idzie streszczenie na strone poczatkowa
\begin{abstract}
  Wysokosc wód podziemnych w gruncie(jakim?) można(jak) opisać równaniem Laplacea w miejscach gdzie woda wychodzi na zewnątrz, tworzą się
  źródła. W okolicy źródła wynika erozja która kształtuje sieć rzeczną
\end{abstract}

\tableofcontents

\chapter*{Wprowadzenie}

Wody podziemne mozna opisać równaniem Laplace'a. Pada deszcz i wynika erozja



\addcontentsline{toc}{chapter}{Wprowadzenie}

\chapter{Model teoretyczny}
\section{Wody podziemne}
\section{Uproszcenie}

\chapter{Rozwiązanie numeryczne i struktura oprogramowania}
\section{Dyzajn i glowne cele oprogromawonia. Szybkość działania vs szybkość prototypowenia.}
\subsection{Geometria}
\subsection{Warunki brzegowe}
\subsection{Generacja meshu}
\subsection{Rozwiazanie}
\subsection{Calkowanie parametrow szeregu}
\subsection{IO i inne}
\section{RiverSim}
\section{pyriversim}

\chapter{Weryfikacja i walidacja oprogramowania}
\section{Zbiezność rozwiazania}
\section{Zbiezność calkowania i porownanie z wynikiem analitycznym}
\section{Optymalizacja predkosci i dokladności}



\chapter{Przyklady obliczen}
\section{Probabilistyczny wzrost rzeki a potegowe prawo.}
\section{Ewolucja zwrotna}
\section{Naczynia krwionosne}

\chapter{Podsumowanie}

\appendix

\chapter{Główna pętla programu}

\chapter{Przykładowe dane wejściowe algorytmu}

\begin{thebibliography}{99}
\addcontentsline{toc}{chapter}{Bibliografia}

\bibitem[Triangle]{shewchuk} Jonathan Richard Shewchuk, \textit{Delaunay Refinement Algorithms for Triangular Mesh Generation, Computational Geometry: Theory and Applications} 22(1-3):21-74, May 2002. PostScript (5,128k, 54 pages).

\bibitem[dealII94]{dealii} Daniel Arndt and Wolfgang Bangerth and Marco Feder and
Marc Fehling and Rene Gassm{\"o}ller and Timo Heister
and Luca Heltai and Martin Kronbichler and
Matthias Maier and Peter Munch and Jean-Paul Pelteret
and Simon Sticko and Bruno Turcksin and David Wells. \textit{The \texttt{deal.II} Library, Version 9.4}.
Journal of Numerical Mathematics. Journal of Numerical Mathematics, vol. 30, no. 3, pages 231-246, 2022.
DOI: 10.1515/jnma-2022-0054.

\bibitem[Tethex]{artemiev} Mikhail Artemiev. \textit{Tethex}. https://github.com/martemyev/tethex.

\bibitem[Boost]{boost} Boost.org

\bibitem[many]{many} Many more.. 

\bibitem[riversim]{riversim} https://github.com/okmechak/RiverSim

\bibitem[riversimpy]{riversimpy} https://github.com/okmechak/RiverSimPY

\end{thebibliography}

\end{document}