\documentclass[]{pracamgr}

\begin{document}

\maketitle

%tu idzie streszczenie na strone poczatkowa
\begin{abstract}
  Wysokosc wód podziemnych w gruncie(jakim?) można(jak) opisać równaniem Laplacea w miejscach gdzie woda wychodzi na zewnątrz, tworzą się
  źródła. W okolicy źródła wynika erozja która kształtuje sieć rzeczną
\end{abstract}

\tableofcontents

\chapter*{Wprowadzenie}
\addcontentsline{toc}{chapter}{Wprowadzenie}

\chapter{Model teoretyczny}
\section{Definicje}
\section{Blabalizator różnicowy}

\chapter{Wcześniejsze implementacje blabalizatora
  różnicowego}
\section{Podejście wprost}
\section{Metody wykorzystujące teorię Głombaskiego}
\section{Metody wykorzystujące własności fetorów $\sigma$}

\chapter{Teoria fetorów $\sigma$-$\rho$}

\chapter{Dokumentacja użytkowa i~opis implementacji}

\chapter{Podsumowanie}

\appendix

\chapter{Główna pętla programu zapisana w~języku T\=oFoo}

\chapter{Przykładowe dane wejściowe algorytmu}

\chapter{Przykładowe wyniki blabalizy
    (ze~współczynnikami~$\sigma$-$\rho$)}

\begin{thebibliography}{99}
\addcontentsline{toc}{chapter}{Bibliografia}

\bibitem[Bea65]{beaman} Juliusz Beaman, \textit{Morbidity of the Jolly
    function}, Mathematica Absurdica, 117 (1965) 338--9.

\end{thebibliography}

\end{document}